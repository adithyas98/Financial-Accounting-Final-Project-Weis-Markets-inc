
\documentclass[12pt]{article}
\usepackage[utf8]{inputenc}
\usepackage[margin=1.0in]{geometry}
\usepackage{amsmath}
\usepackage{setspace}
\title{Financial Accounting Final Project:\\Weis Markets, Inc.}
\author{Adithya Shastry, Erik Tacam, Brian Long}
\date{December 4, 2018}

% for double spacing use the following model

%    \begin{doublespacing}
%    Your text. 
 %   \end{doublespacing}



\begin{document}

\maketitle   
\newpage
\newpage
\textbf{1 Abstract}
\\
\begin{doublespacing}
Weis Markets is a small cap retail food company located in the Northeasters region of the United States. Analysis of the business operations of the company found very few growth prospects for the company because of a very highly competitive market and poor decisions by the management. Analysis and comparison of Weis Markets with its direct competitors found that the company's key financial ratios are common with both a well performing company and a poorly performing company.Finally,analysis of the financial statements further proved that the company has very few growth prospects as well and that there may be some aggressive accounting standards used by the company to better present their company to current and future shareholders.
\\

\textbf{2 Analysis of Business Model}
\\
Weis Markets is primarily engaged in the retail sale of food and food products in the Northeast region of the United States. Since the company is in the retail food industry , it has many competitors such as Sprouts and Natural Grocers by Vitamin Cottage. The relevant risk factors were discussed in the Management Discussion and Analysis section of the 10-K report for the company. In the risk factors section of the Management Discussion and Analysis section, the company mentioned some very common factors of risk such as economic downturn, weather, public opinion, and other things. All of the factors mentioned in this section of the 10-K report seemed to be ordinary with regards to the type of business Weis Markets is, with the exception of a few risks. For example, the risk of the competitiveness of the the industry and the need to keep up with the demands of the customers. Although, this is a normal risk for a food retail company like Weis Markets, as mentioned before, this is a very important factor to consider when evaluating the growth of the company in the future. Historically, the performance of a company in an ultra competitive market has always been determined by the management of the company at the highest levels. Initial review of the executive officers of the registrant section, showed that most of the chief executives of the company have very relevant experiences with relation to their current jobs at Weis Markets. 

Initial interest in Weis Markets was derived from the company’s stock undergoing a 52-week low. At the surface, the company seems to be charged with aggressive accounting by the Securities Exchange Commission. Analysis of these charges will be done in a later section labeled "Analysis of Aggressive Accounting Charges."

The potential growth of the business can be analyzed through the investments the company is making into products, trademarks, stores, and other assets. First, let us analyze the company's investments in stores. We can  see in the 2017 10-K report for Weis Markets that the company has not made many changes to the number of stores they have. For example, in 2017 the company had a net addition of one store.\footnote{“WMK 2017 10-K.” Weis Markets, 2018, weismarkets.com.}

This shows that the company is not trying to expand business through the opening of new stores. Although this is a very costly method for expanding business, at least in the short run, it will provide expanded revenues well into the foreseeable future if done correctly. Given the company's revenue and cash flow increases, it is strange that the company is not investing in expanding their business to more areas, and as a result more costumers. 

We can now consider the company's investments into new products, by considering the amount of money spent on Research and Development and by looking into intangible assets the company has on its balance sheet. First, we can analyze the company's purchases of intangible assets. From the 2017 10-K report for the Weis Markets, we can see that the balance sheet shows that the number of intangible assets has increased by 17 percent. While there is an increase in the overall value of the intangible assets for the company, this is a normal change for the company when considering the cash outflows for the company for the purchase of intangible assets. Given this fact and the revenue growth that the company endured between 2015 and 2016, it is clear that the increase in the intangible assets for the company is not the factor that is contributing to its growth.

Next, we can look into internal research and development which, because of accounting rules, will not be reported on the balance sheet. We can instead use NASDAQ. NASDAQ points out that after their research into the firm, they have concluded that the company does not put any money into research and development.\footnote{“WMK Income Statement.” NASDAQ.com, NASDAQ, www.nasdaq.com/symbol/wmk/financials?query=income-statement.} This is a very glaring red flag because it means the company is not using in house researchers to develop new products for the company. Products that may give the company a competitive advantage over other companies that occupy the same market space. Since, this is a very competitive industry, it is very important from an investor's perspective that a company have a competitive advantage, or moat, before investing in that company. A proper competitive advantage will ensure that the company has the ability grow in the future by gaining a higher market share.\footnote{Graham, Benjamin, et al. The Intelligent Investor: a Book of Practical Counsel. Harper Collins, 2013.}

Let us now consider the changes in store locations over the past three years. The company details its retail store changes in item 1 of the 10k report\footnote{“WMK 2017 10-K.” Weis Markets, 2018, weismarkets.com.}.This table reports many figures including store closes, re modelings, and new store openings. We can safely assume that the change in the building of new stores will have the greatest impact on the overall growth of the company. This is a safe assumption because building a new store in a new area has the possibility to attract more costumers than a renovation of a store. We can see that the number of stores added is fairly consistent, except for a jump to 44 stores in 2016. These acquisitions resulted in a 11.9 percent increase in revenues. This is a good sign, but the trend does not continue in 2017 where only two new stores were built. This could mean that either there is something abnormal happening with the revenue recognized (this would be consistent with the aggressive accounting charges) or that the growth from those new stores is not repeatable. Either way, this is not a good sign for an investor since there is something hidden about the revenue growth. 

Given all of this information it is quite clear that from an investor's perspective this company may not be a good investment because the growth prospects of the company are limited. However, further analysis of key ratios over the course of a few years, Weis Market's performance compared to its direct competitors, and the aggressive accounting will give a complete picture of the company and a more accurate view of the performance of the company.
\\

\textbf{3 Three Year Time-Series Analysis}
\\
In order to better understand whether Weis Markets is a great company to invest in, one must first perform a deep analysis of the company's performance over the past years. As time passes by, investors want to see organic growth within a company. Long-term investors want to be able to see a company grow overtime and yield better finances. Unfortunately for these investors, Weis Markets does not appear to be a company that they would want to invest in. Over the past three years, Weis Markets has been under-performing. The company has failed to improve its finances and instead has decreased it competitiveness.
One of the first few things that everyone looks for in a company is the amount of cash that it has available. Cash is a current asset that allows for a company to be able to make purchases that will hopefully yield high revenue and also pay its debt. Taking a look into Weis Markets’ cash and cash equivalents over the past three years, we can see that the company has had an inconsistent amount of cash as reported in its balance sheet. In 2015, the company had \$17,596 (in thousands) in cash and cash equivalents. This amount decreased by 17 percent in 2016, which was \$14,653. This amount then increased by 55 percentage in 2017. This is a large growth in cash and cash equivalents for 2017, leading to an amount of \$21,305.\footnote{“WMK 2017 10-K.” Weis Markets, 2018, weismarkets.com.} This is by far the largest amount that the company has had in the past three years, however, these are just simply numbers. Every company has some amount of cash in balance sheets. What matters the most is whether the cash is sufficient for the company to be sustainable and not rely on increasing debt. In order to better understand these cash and cash equivalents numbers, we take a look into cash ratio. Cash ratio measures the amount of cash and cash equivalents a company has over current liabilities. This ratio shows whether the company has enough cash to be able to pay for its current liabilities. Unfortunately for Weis Markets, the company did not have enough cash to pay for its current liabilities over the past 3 years and their cash ratio has decreased drastically since 2015. In 2015 its cash ratio was 53 percent, with that dropping drastically to 34 percent in 2016. In 2017 the company had a slight boost to 36 percent.\footnote{“WMK Income Statement.” NASDAQ.com, NASDAQ, www.nasdaq.com/symbol/wmk/financials?query=ratios.} Even though the company had the largest amount of cash in 2017, its cash ratio was significantly worse when compared to 2015. This means that the company has been largely increasing its current liabilities without having a linear or proportionate or exponential growth to its cash balance.
However, current liabilities do not have to necessarily be paid in cash. Weis Markets could choose to sell off their assets to be able to generate enough revenue to pay for its current liabilities. In order to analyze whether the company has enough assets to be able to pay off their current liabilities, we have to take a deeper look into the company’s current assets and current liabilities. By having these two measurements, we are then able to measure the company’s ability to pay short-term obligations by finding its current ratio. The current ratio is measured by taking the current assets over the current liabilities. For Weis Markets, these numbers are significantly good for the company. In 2015, the company had a current ratio of 2.05. This meant that the company was able to pay off double the amount of its current liabilities with its current assets. This number has since decreased to 1.75 in 2016 and 1.76 in 2017.\footnote{“WMK Income Statement.” NASDAQ.com, NASDAQ, www.nasdaq.com/symbol/wmk/financials?query=ratios.} These ratios still mean that the company is able to pay off its current liabilities with current assets, however the company is not able to maintain a stable or increasing current ratio.
Another ratio that’s important to look at when considering whether the company is able to meet its current liabilities is quick ratio. Quick ratio measures how well a company can meet its short-term financial liabilities. This is measured by cash plus marketable securities plus accounts receivable all over current liabilities. In 2015 the company had a quick ratio of 1.01, meaning that it was barely able to pay off its current liabilities without selling any current assets. In 2016, this ratio dropped by 26 percent to 0.75. In 2017, it continued to drop to 0.74.\footnote{“WMK Income Statement.” NASDAQ.com, NASDAQ, www.nasdaq.com/symbol/wmk/financials?query=ratios.} The quick ratio for Weis Markets over the past 3 years shows that the company is having difficulties in being able to pay off its current liabilities. Although the company does have enough revenue to pay off its current liabilities if it sells off its current liabilities, the company would not exist if it would have to constantly sell off its stores and other assets. In reality, the company pushes its current liabilities for a year longer to worry about it in the near future rather than in the present.
Why is it that the company is not able to generate enough money to pay off its current debts? Well the answer might be found by taking a deep look into the company’s gross margin and operating margin. Gross margin measures the difference between revenue and COGS divided by revenue. This allows for investors to measure whether the company is able to be profitable by maintaining COGS expense low. For the past 3 years, the gross margin for Weis Markets has been consistent between 27 percent and 28 percent.\footnote{“WMK Income Statement.” NASDAQ.com, NASDAQ, www.nasdaq.com/symbol/wmk/financials?query=ratios.} This means that between 72 percent and 73 percent of revenue is spent on COGS expense, which ultimately affects the company’s ability to generate a large net income and be able to pay off its current liabilities when it has so many COGS expenses to pay. Weis had a COGS expense of \$2,090,016 in 2015, \$2,264,565 in 2016, and \$2,540,348 in 2017, in which the expense increased every year.\footnote{“WMK 2017 10-K.” Weis Markets, 2018, weismarkets.com.} Another expense that the company has to make is SG\&A expense. Weis had an SG&A expense of \$695,953 in 2015,\$773,830 in 2016, and \$850,034 in 2017, in which the expense increased every year.\footnote{“WMK 2017 10-K.” Weis Markets, 2018, weismarkets.com.} To calculate for the impact that SG\&A has on revenue, we use the operating margin. This margin is the ratio of operating income to net sales, measured by sales subtract COGS and SG\&A all over sales. The operating margin for Weis Markets has been consistent between 2 percent and 3 percent over the past 3 years and show that the company is having difficulty in limiting its expenses.\footnote{“WMK Income Statement.” NASDAQ.com, NASDAQ, www.nasdaq.com/symbol/wmk/financials?query=ratios.} This margin shows that for over the past 3 years Weis Markets has been spending over 95 percent of its revenue on expenses. This limits the company’s ability to increase its profits and be able to pay off its current liabilities and also long-term debt.
Since the company has low gross margin and operating margin, the company consequently has a low profit margin. Profit margin measures the amount by which revenue from sales exceeds costs. Over the past three years, this margin has been consistent between 2 percent and 3 percent.\footnote{“WMK Income Statement.” NASDAQ.com, NASDAQ, https://www.nasdaq.com/symbol/wmk/financials?query=ratios.} This means that the company is barely able to increase its income and therefore unable to generate enough money to pay off its liabilities. A 3 percent growth is not ideal for long-term investors, but it is by no means an automatic disqualification in choosing to invest in the company. Long-term investors take into account how competitive the company is compared to its main competitors, which is discussed in the section below.
\\

\textbf{4 Cross-Sectional Analysis}
\\
We now compare Weis Markets to two of its competitors, Sprouts Farmers Markets and Natural Grocers. We report data for all three companies for years 2016 and 2017. On our data chart, we referred to these years as years 2 and 3 of our 3 year analysis of Weis Markets. To avoid confusion, we now refer to these years as years 1 and 2 for our two year comparison of all three firms. All numerical data is reported in thousands. During this 2 year period, Weis Markets saw their annual revenues increase from \$3,136,720 to \$3,466,807, a 11\% increase.\footnote{“WMK 2017 10-K.” Weis Markets, 2018, weismarkets.com.} Comparatively, Sprouts saw an increase of 15\%, from \$4,066,385 to \$4,664,612, while Natural Grocers saw an increase of 9\%, from\$ 705,499 to\$ 769,030.\footnote{SFM Company Financials. (2018, 12 3). Retrieved December 4, 2018, from Nasdaq website: 
     https://www.nasdaq.com/symbol/sfm/real-time }\footnote{NGVC Company Financials. (2018, 12 3). Retrieved December 4, 2018, from Nasdaq website: 
     \footnote{SFM Company Financials. (2018, 12 3). Retrieved December 4, 2018, from Nasdaq 
     website: https://www.nasdaq.com/symbol/sfm/real-time } } Weis Markets reported gross margin percentages of 28\% and 27\% for years 1 and 2 while Sprouts reported 29\% for both years and Natural reported 29\% and 28\%.\footnote{“WMK 2017 10-K.” Weis Markets, 2018, weismarkets.com.}\footnote{SFM Company Financials. (2018, 12 3). Retrieved December 4, 2018, from Nasdaq website: 
     https://www.nasdaq.com/symbol/sfm/real-time }\footnote{NGVC Company Financials. (2018, 12 3). Retrieved December 4, 2018, from Nasdaq website: 
     \footnote{SFM Company Financials. (2018, 12 3). Retrieved December 4, 2018, from Nasdaq 
     website: https://www.nasdaq.com/symbol/sfm/real-time } } All three companies reported more or less the same percentages for gross margin. We recall that gross margin is a measure of percentage revenues retained after all COGS expenses have been paid. Effectively, this is a measure of how much the firms each make from selling groceries after recognizing the expenses they incurred to obtain the groceries in the first place. Unsurprisingly, the firms all reported very similar, nearly identical levels of gross margin. This means that all firms tend to spend more or less the same amount to obtain their products when compared to their revenues earned. Being that the market is competitive, it is likely that all three sell their products at very similar levels and purchase their products from suppliers at very similar price levels. The differences in numerical value between all gross margins and revenues depends simply on the size and market share of each firm. Since Sprouts reported the highest annual revenues, we can infer that Sprouts controls a larger market share, or equivalently gets more business, than Weis or Natural. 

Further, Weis Markets reported a 3\% operating profit margin for year one and 2\% for year 2, while Sprouts reported 5\% for both years.\footnote{“WMK 2017 10-K.” Weis Markets, 2018, weismarkets.com.} Natural Grocers, however, dropped from 3\% to 2\%. This means that, before paying interest expenses and income tax, Sprouts retained 5\% of their overall revenues, while Natural retained 3\% and subsequently 2\% the following year.\footnote{SFM Company Financials. (2018, 12 3). Retrieved December 4, 2018, from Nasdaq website: 
     https://www.nasdaq.com/symbol/sfm/real-time }\footnote{NGVC Company Financials. (2018, 12 3). Retrieved December 4, 2018, from Nasdaq website: 
     \footnote{SFM Company Financials. (2018, 12 3). Retrieved December 4, 2018, from Nasdaq 
     website: https://www.nasdaq.com/symbol/sfm/real-time } } From this we can see that Sprouts Markets incurs the least depreciation/amortization and SG & A expenses relative to their revenues. Weis and Natural Grocers both incur similar percentage costs relative to their revenues. Comparing Sprouts' and Natural's annual revenues, this result is expected. Since Natural makes markedly less annual revenues, certain fixed costs, such as utilities, will encompass a larger portion of their annual earnings. The fact that Natural and Weis both incur similar cost percentages, despite Weis having a markedly higher annual revenue, implies that Weis may be incurring more annual costs than its competitors, or that they may be overvaluing their revenues. 

Weis Markets reported a current ratio of 1.75 and 1.8 for years 1 and 2 respectively.\footnote{“WMK 2017 10-K.” Weis Markets, 2018, weismarkets.com.} Conversely, Sprouts reported 1.02 and 1 while Natural reported 1.46 and 1.51.\footnote{SFM Company Financials. (2018, 12 3). Retrieved December 4, 2018, from Nasdaq website: 
     https://www.nasdaq.com/symbol/sfm/real-time }\footnote{NGVC Company Financials. (2018, 12 3). Retrieved December 4, 2018, from Nasdaq website: 
     \footnote{SFM Company Financials. (2018, 12 3). Retrieved December 4, 2018, from Nasdaq 
     website: https://www.nasdaq.com/symbol/sfm/real-time } } The current ratio is a measure of how easily a company can pay off it’s outstanding current liabilities using current assets. Of the three firms, Sprouts is the least equipped to accomplish this. In the second year they reported a current ratio of 1, implying that they have \$1 worth of assets to pay of each \$1 worth of liabilities. That is, if they were forced to pay off all liabilities, it would cost them all of their assets. In contrast, Weis markets reported a current ratio of 1.75 and 1.8 across the two year period, implying that for each dollar’s worth of liabilities, they had \$1.75 and \$1.8 worth of with which to pay. However, many of these assets are not in cash or a cash equivalent form, they are simply represented by the minimum of their market value and acquisition cost. This means that, if a debt were called in immediately, many of these assets would be useless in paying it off. The quick ratio solves this issue as it reports only assets which can be liquidated within 30 days. Weis markets reported a quick ratio of .74 for year 2 and .75 for year 1.\footnote{“WMK Income Statement.” NASDAQ.com, NASDAQ, www.nasdaq.com/symbol/wmk/financials?query=income-statement.} This means that for the second year, only .74 of each \$1 worth of Weis Markets’ liabilities can be paid off within 30 days. Sprouts, however, reported a quick ratio of .23 for both years while Natural Grocers reported .17 for year 1 and .2 for year 2. Thus, it follows that, of the three competing firms, Weis markets is the best equipped to pay off their outstanding liabilities. Since the quick ratio is less than 1, all three firms are ill-equipped to pay off their current outstanding debt within 30 days. To combat this, the firms may choose to liquidate some of their long term assets to cash, or they may put inventory on sale to help convert inventory to cash faster. 

Weis reported and asset turnover ratio of 2.35 and 2.41 for years 1 and 2 respectively.\footnote{“WMK Income Statement.” NASDAQ.com, NASDAQ, www.nasdaq.com/symbol/wmk/financials?query=income-statement.} Sprouts reported a growth from 2.84 to 3.09 while Natural Grocers reported a fall from 2.73 to 2.64.\footnote{SFM Company Financials. (2018, 12 3). Retrieved December 4, 2018, from NASDAQ website: https://www.nasdaq.com/symbol/sfm/real-time } The asset turnover ratio effectively measures the efficiency with which a firm can convert assets into revenues. That is, Weis was able to generate \$2.41 in revenues for each \$1 spent on assets. The above result suggests that Sprouts is both the most efficient and most improved of the three firms in converting assets into revenues while Weis is the least efficient for both years. Weis Markets did show a slight upward trend during the two year period, which may imply future improvement. Sprouts, however, showed the highest increase of the three firms, improving their efficiency by .25. Further, Weis reported an inventory turnover ratio of 8.95 and 9.13 for years 1 and 2.\footnote{“WMK Income Statement.” NASDAQ.com, NASDAQ, www.nasdaq.com/symbol/wmk/financials?query=income-statement.} Sprouts reported 15.39 and 15.28 while Natural reported 6.25 and 6.19.\footnote{SFM Company Financials. (2018, 12 3). Retrieved December 4, 2018, from Nasdaq website: 
     https://www.nasdaq.com/symbol/sfm/real-time }\footnote{NGVC Company Financials. (2018, 12 3). Retrieved December 4, 2018, from Nasdaq website: 
     \footnote{SFM Company Financials. (2018, 12 3). Retrieved December 4, 2018, from Nasdaq 
     website: https://www.nasdaq.com/symbol/sfm/real-time } } The inventory turnover ratio is a measure of how efficiently a given firm manages and sells its inventory. Put simply, the measure gives how many times a firm sells or “turns over” its average inventory. Thus, a company with a high inventory turnover ratio is selling their inventory at a high rate. Since Weis reported an inventory turnover of 9.13, it follows that they were able to sell their average inventory level 9.13 times over. Of the three, Sprouts proved to be the most efficient in its inventory sale, but Weis markets was the only firm which was able to improve its inventory turnover during the 2 year period. Further, Weis reported an accounts receivable turnover ratio of 33.44 and 37.95 for years 1 and 2 respectively\footnote{“WMK Income Statement.” NASDAQ.com, NASDAQ, www.nasdaq.com/symbol/wmk/financials?query=income-statement.}. Sprouts reported 177.27 and 182.49 while Natural Grocers reported 196.27 and 178.7\footnote{SFM Company Financials. (2018, 12 3). Retrieved December 4, 2018, from Nasdaq website: 
     https://www.nasdaq.com/symbol/sfm/real-time }\footnote{NGVC Company Financials. (2018, 12 3). Retrieved December 4, 2018, from Nasdaq website: 
     \footnote{SFM Company Financials. (2018, 12 3). Retrieved December 4, 2018, from Nasdaq 
     website: https://www.nasdaq.com/symbol/sfm/real-time } } Much like the inventory turnover ratio, the accounts receivable turnover ratio measures how efficiently a firm is able to collect cash from outstanding receivables from debt. That is, a firm with a high accounts receivable turnover ratio is desirable as it is able to liquidate its outstanding receivable from debt much faster than a firm with a low turnover rate. Of the 3 firms, Weis reported by far the lowest accounts receivable turnover ratios, reporting 37.95 in year as compared to Sprouts’ 182.49 and Natural’s 178.7. This implies that Weis is the least efficient in terms of their debt collection, increasing the length of Weis’ operating cycle. From these two measures, it follows that Weis had an operating cycle of 51.7 days for year 1 and 49.6 days for year 2. Natural Grocers had an overall operating cycle of 60.3 days for year 1 and 61 days for year 2 while Sprouts had an operating cycle of 25.8 days for year 1 and 25.9 days for year 2. The operating cycle represents the total amount of days required for a firm to sell its inventory and subsequently convert said inventory to cash or a cash equivalent. Since Sprouts had the highest inventory turnover and accounts receivable turnover ratios for both years, it is unsurprising that it also had the shortest operating cycle. The majority of all three firms’ operating cycles can be attributed to their inventory turnover ratios. That is, because the accounts receivable turnover ratios were all comparatively much larger than the inventory turnover ratios, it follows that the majority of the operating cycle is spent selling off inventory rather than collecting debt. Since the wide majority of customers pay up front for groceries, this result is unsurprising. 

Weis reported a total return on equity (ROE) percentage of 10\% for both year one and year two respectively.\footnote{“WMK 2017 10-K.” Weis Markets, 2018, weismarkets.com.} Conversely, Sprouts recorded an ROE of 16\% for year one and 21\% for year two while Natural Grocers recorded an ROE of 12\% for year one and 6\% for year two.\footnote{SFM Company Financials. (2018, 12 3). Retrieved December 4, 2018, from Nasdaq website: 
     https://www.nasdaq.com/symbol/sfm/real-time }\footnote{NGVC Company Financials. (2018, 12 3). Retrieved December 4, 2018, from Nasdaq website: 
     \footnote{SFM Company Financials. (2018, 12 3). Retrieved December 4, 2018, from Nasdaq 
     website: https://www.nasdaq.com/symbol/sfm/real-time } } For investors, a high return on equity ratio is desirable, it can be used as a measure to how much investors benefit from funding a given firm. Sprouts recorded by far the highest ROE. Conversely, despite having similarly high levels of revenue, Weis markets recorded markedly lower ROE, far closer to Natural Grocers. Further, Weis recorded a return on invested capital (ROIC) percentage of 9\% for year one and 6\% for year two.\footnote{“WMK 2017 10-K.” Weis Markets, 2018, weismarkets.com.} Once again Sprouts recorded the highest returns at 12\% for year one and 16\% for year two while Natural Grocers recorded 7\% for year one and 6\% for year two.\footnote{SFM Company Financials. (2018, 12 3). Retrieved December 4, 2018, from Nasdaq website: 
     https://www.nasdaq.com/symbol/sfm/real-time }\footnote{NGVC Company Financials. (2018, 12 3). Retrieved December 4, 2018, from Nasdaq website: 
     \footnote{SFM Company Financials. (2018, 12 3). Retrieved December 4, 2018, from Nasdaq 
     website: https://www.nasdaq.com/symbol/sfm/real-time } } Return on Invested Capital measures the efficiency with which a firm is able to generate income from an investment. A high ROIC is desirable, it implies that a firm is able to generate a higher amount of income per dollar of capital invested. As was the case for return on equity, Weis markets reported ROIC levels similar to those of Natural Grocers, despite recording much higher levels of annual revenues. Sprouts, on the other hand, recorded far higher levels of ROIC. 

From the aforementioned results, a disparity in the data becomes evident. Based off of Weis Markets’ reported annual revenues, as well as all measures derived from revenues, one may assume that Weis Markets generates a significant amount of business from their grocery sales. Similarly, when considering their annual revenues, one may also assume that Natural Grocers generates a significantly smaller level of Grocery business when compared to Sprouts and Weis. As mentioned in the analysis of annual revenues, the three firms are in a competitive market, and thus it was assumed that they sell their products at similar price levels. However, recent price checks for a market basket of items showed that Weis Markets charged up to 15\% more than their peers on goods\footnote{S. (2018). Weis Markets, Inc (pp. 1-46, Rep.). New York, New York: Spruce Point Capital.}. This may explain why their asset turnover ratio was that much lower than Sprouts’ despite reporting over \$3 billion in annual revenues. Through aggressive accounting, to be better defined and analyzed below, Weis Markets was able to overvalue their annual revenues. Hence, those measures which do not involve their recorded revenues, such as the return ratios and operating cycle, appear to be at a similar level to those of Natural Grocers, a far smaller and less successful Grocery business. 
\\

\textbf{5 Analysis of Aggressive Accounting Charges}
\\
As mentioned previously, Weis Markets has been allegedly following aggressive accounting standards in order to make their company look better to investors. After some review of the literature on this topic we have found a couple of key things to look for when it comes to finding the extent to which the company is actually practicing aggressive accounting and to what extent that affects our view of the company from a investor's point of view. Some important factors we focused on were how the company recognized revenue, the store's price points compared to their immediate competitors, the amount of stocks owned by the management, changes in the auditing company, and their transparency with their investors and general public. 
\\
First, we will examine the methodology the company uses to recognize revenue. This can be found in the Revenue Recognition section of Note 1:Summary of Significant accounting policies. In this section the company points out very standard revenue recognition standards with a few exceptions. When it comes to their customer rewards program, which is a standard program that many retail companies use to keep customers coming back to their stores, Weis Markets states that they recognize future discounts given to customers by making "reasonable and reliable estimates over the life of the program" \footnote{“WMK 2017 10-K.” Weis Markets, 2018, weismarkets.com.}. This is a very interesting way to conduct their recognition of these types of sales given their ability to recognize reductions in sale prices for their preferred shopper loyalty program. This could be problem from an investor's perspective if a large percentage of the company's revenue comes from one of these programs since the company can essentially "estimate" the revenue as high or low as they want when reporting it on the financial statements. Given this possibility, it is also very concerning that the company does not provide any numbers on the percentage of sales that are done under these types of programs\footnote{“WMK 2017 10-K.” Weis Markets, 2018, weismarkets.com.}.
\\
Next, we will analyze the price points for Weis markets and how they compare to the price points for their competitors. After reading an article by a research firm that was hired by the shareholders of Weis Markets in order to analyze the extent to which the company was practicing aggressive accounting standards, we found that the price points for the business are much above those of its competitors. While this might be fine in other industries, it is certainly not going to work in the highly competitive grocery industry. When the research firm analyzed their price points, they found that the price points at Weis Markets were for some products up to 20 percent higher than their competitors for the same product. For example, the price for cheerios cereal was at an astonishing 47 percent premium when compared to their competitors\footnote{S. (2018). Weis Markets, Inc (pp. 1-46, Rep.). New York, New York: Spruce Point Capital.}. Given this, it is very difficult to see how Weis Markets could have such impressive growth. 

The next fact about the company to analyze is how much of the company's equity is given to the management,either in the form of full stocks or stock options. According to the research done by Spruce Point Capital, the company's executives have very little stake in the company. This is clearly stated by the company in a press release where the company stated that "the compensation committee does not believe that equity-based incentives are a valuable incentive for the employees of the company"\footnote{S. (2018). Weis Markets, Inc (pp. 1-46, Rep.). New York, New York: Spruce Point Capital.}. This is a very dangerous thing for the company to say from the investor's perspective of the company because it means the company's executives aren't monetarily motivated to act in the benefit of the shareholders. It may also mean that the management does not believe in the company enough to take stock in the company as a part of their compensation plan. This is equally as terrifying as the last point and therefore should serve as a red flag for investors. This sentiment is also shown on the glass door page for the company where only 31 percent of the employees at the firm have a positive outlook on the business\footnote{S. (2018). Weis Markets, Inc (pp. 1-46, Rep.). New York, New York: Spruce Point Capital.}.This goes along with the previous two points and should be seen as a warning for investors for those reasons.
\\
Finally, we have the very serious issue of how the company handles its investor relations and the companies that audit its financial statements. The report from Spruce Point Capital shows that the company has had to change the auditing company three times in the last four years. This is a very glaring sign that something a little illegitimate is going on when it comes to preparing and presenting the financial statements for the company. This again should serve as a very glaring red flag for an investor because it calls into question the validity of the financial statements provided by the company. 

Overall, through the analysis of all of these factors we can see that the company is following some very aggressive accounting standards in order to cover up some financial difficulties that the company is facing. It is clear that the company might be facing some very serious financial trouble given the pricing model used by the company and absence of any substantial organic growth as mentioned in the Analysis of Business Model section of the paper.
\\

\textbf{6 Conclusion}
\\
After analyzing all of the Weis Market's financial statements and comparing key ratios to its competitors, we have found Weis Markets to not be a sound investment.
From a cross sectional analysis, it follows that Weis Markets reports similar revenue levels, as well as all revenue-derived measures, to those of Sprouts Markets. Conversely, their return ratios, ROIC and ROE, are far lower than those of Sprouts, reaching similar levels to those of Natural Grocers. Based off of their annual revenues, it can be inferred that Sprouts Farmers Market controls a considerably larger market share than Natural Grocers. This would explain Sprouts' higher revenues, higher return ratios, and shorter operating cycle. Because of their higher prices, Weis Markets has much lower returns as well as a much longer operating cycle than one may expect based off of their revenues. Analysis of Weis Markets' accounting policies and further analysis of a report by Spruce Point Capital supported the alleged aggressive accounting claims standards that the shareholders of the company had found. The company seems to be undergoing some financial trouble which can be explained by their premium prices when compared to other competitors in the industry. But as a small company in which large amounts of growth are mandated, the company has elected to obfuscate some of their financial reports in order to show more value to current and future shareholders. 




\end{doublespacing}
\end{document}
